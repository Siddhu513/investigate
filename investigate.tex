% Options for packages loaded elsewhere
\PassOptionsToPackage{unicode}{hyperref}
\PassOptionsToPackage{hyphens}{url}
%
\documentclass[
]{article}
\usepackage{amsmath,amssymb}
\usepackage{lmodern}
\usepackage{iftex}
\ifPDFTeX
  \usepackage[T1]{fontenc}
  \usepackage[utf8]{inputenc}
  \usepackage{textcomp} % provide euro and other symbols
\else % if luatex or xetex
  \usepackage{unicode-math}
  \defaultfontfeatures{Scale=MatchLowercase}
  \defaultfontfeatures[\rmfamily]{Ligatures=TeX,Scale=1}
\fi
% Use upquote if available, for straight quotes in verbatim environments
\IfFileExists{upquote.sty}{\usepackage{upquote}}{}
\IfFileExists{microtype.sty}{% use microtype if available
  \usepackage[]{microtype}
  \UseMicrotypeSet[protrusion]{basicmath} % disable protrusion for tt fonts
}{}
\makeatletter
\@ifundefined{KOMAClassName}{% if non-KOMA class
  \IfFileExists{parskip.sty}{%
    \usepackage{parskip}
  }{% else
    \setlength{\parindent}{0pt}
    \setlength{\parskip}{6pt plus 2pt minus 1pt}}
}{% if KOMA class
  \KOMAoptions{parskip=half}}
\makeatother
\usepackage{xcolor}
\usepackage[margin=1in]{geometry}
\usepackage{graphicx}
\makeatletter
\def\maxwidth{\ifdim\Gin@nat@width>\linewidth\linewidth\else\Gin@nat@width\fi}
\def\maxheight{\ifdim\Gin@nat@height>\textheight\textheight\else\Gin@nat@height\fi}
\makeatother
% Scale images if necessary, so that they will not overflow the page
% margins by default, and it is still possible to overwrite the defaults
% using explicit options in \includegraphics[width, height, ...]{}
\setkeys{Gin}{width=\maxwidth,height=\maxheight,keepaspectratio}
% Set default figure placement to htbp
\makeatletter
\def\fps@figure{htbp}
\makeatother
\setlength{\emergencystretch}{3em} % prevent overfull lines
\providecommand{\tightlist}{%
  \setlength{\itemsep}{0pt}\setlength{\parskip}{0pt}}
\setcounter{secnumdepth}{-\maxdimen} % remove section numbering
\ifLuaTeX
  \usepackage{selnolig}  % disable illegal ligatures
\fi
\IfFileExists{bookmark.sty}{\usepackage{bookmark}}{\usepackage{hyperref}}
\IfFileExists{xurl.sty}{\usepackage{xurl}}{} % add URL line breaks if available
\urlstyle{same} % disable monospaced font for URLs
\hypersetup{
  hidelinks,
  pdfcreator={LaTeX via pandoc}}

\author{}
\date{\vspace{-2.5em}}

\begin{document}

\hypertarget{part-1-simulation-exercise}{%
\section{Part 1: Simulation Exercise}\label{part-1-simulation-exercise}}

\hypertarget{overview-this-part-is-going-to-execute-simulations-and-data-analysises-to-illustrate-application-of-the-central-limit-theorem.-r-programming-will-be-the-major-tool-to-realize-the-mentioned-goal.}{%
\section{Overview: This part is going to execute simulations and data
analysises to illustrate application of the central limit theorem. R
programming will be the major tool to realize the mentioned
goal.}\label{overview-this-part-is-going-to-execute-simulations-and-data-analysises-to-illustrate-application-of-the-central-limit-theorem.-r-programming-will-be-the-major-tool-to-realize-the-mentioned-goal.}}

\hypertarget{question-1-show-the-sample-mean-and-compare-it-to-the-theoretical-mean-of-the-distribution.}{%
\subsection{Question 1: Show the sample mean and compare it to the
theoretical mean of the
distribution.}\label{question-1-show-the-sample-mean-and-compare-it-to-the-theoretical-mean-of-the-distribution.}}

knitr::opts\_chunk\$set(echo = TRUE) lambda \textless- 0.2 simData
\textless- matrix(rexp(1000*40, lambda), nrow = 1000, ncol = 40)
distMean \textless- apply(simData, 1, mean) hist(distMean, breaks = 50,
main = ``The distribution of 1000 averages of 40 random exponentials'',
xlab = ``Value of means'', ylab = ``Frequency of means'', col =
``pink'') abline(v = 1/lambda, lty = 1, lwd = 5, col = ``red'')
legend(``topright'', lty = 1, lwd = 5, col = ``red'', legend =
``theoretical mean'')

\#The simulated sample means are normally distributed with a center very
close to the theoretical mean.

\hypertarget{question-2-show-how-variable-the-sample-is-via-variance-and-compare-it-to-the-theoretical-variance-of-the-distribution.}{%
\subsection{Question 2: Show how variable the sample is (via variance)
and compare it to the theoretical variance of the
distribution.}\label{question-2-show-how-variable-the-sample-is-via-variance-and-compare-it-to-the-theoretical-variance-of-the-distribution.}}

distVar \textless- apply(simData, 1, var) hist(distVar, breaks = 50,
main = ``The distribution of 1000 variance of 40 random exponentials'',
xlab = ``Value of variances'', ylab = ``Frequency of variance'', col =
``light blue'') abline(v = (1/lambda)\^{}2, lty = 1, lwd = 5, col =
``blue'') legend(``topright'', lty = 1, lwd = 5, col = ``blue'', legend
= ``theoretical variance'')

\#The simulated sample variances are almost normally distributed with a
center near the theoretical variance.

\hypertarget{question-3-show-that-the-distribution-is-approximately-normal.}{%
\subsection{Question 3: Show that the distribution is approximately
normal.}\label{question-3-show-that-the-distribution-is-approximately-normal.}}

par(mfrow = c(3, 1)) hist(simData, breaks = 50, main = ``Distribution of
exponentials with lambda equals to 0.2'', xlab = ``Exponentials'', col =
``yellow'') hist(distMean, breaks = 50, main = ``The distribution of
1000 averages of 40 random exponentials'', xlab = ``Value of means'',
ylab = ``Frequency of means'', col = ``pink'') simNorm \textless-
rnorm(1000, mean = mean(distMean), sd = sd(distMean)) hist(simNorm,
breaks = 50, main = ``A normal distribution with theoretical mean and sd
of the exponentials'', xlab = ``Normal variables'', col = ``light
green'')

\hypertarget{results}{%
\subsubsection{Results}\label{results}}

\hypertarget{the-first-histogram-is-the-distribution-of-the-exponentials-with-lambda-equals-to-0.2.-the-second-histogram-is-the-distribution-of-1000-averages-of-40-random-exponentials}{%
\paragraph{1. The first histogram is the distribution of the
exponentials with lambda equals to 0.2. The second histogram is the
distribution of 1000 averages of 40 random
exponentials}\label{the-first-histogram-is-the-distribution-of-the-exponentials-with-lambda-equals-to-0.2.-the-second-histogram-is-the-distribution-of-1000-averages-of-40-random-exponentials}}

\hypertarget{the-third-histogram-is-a-real-normal-distribution-with-a-mean-and-standard-deviation-equals-to-the-second-histograms.}{%
\paragraph{2. The third histogram is a real normal distribution with a
mean and standard deviation equals to the second
histogram's.}\label{the-third-histogram-is-a-real-normal-distribution-with-a-mean-and-standard-deviation-equals-to-the-second-histograms.}}

\hypertarget{comparing-the-first-with-the-second-histogram-we-can-see-the-distrubution-becames-normal-as-the-means-were-taken-from-each-groups.}{%
\paragraph{3. Comparing the first with the second histogram, we can see
the distrubution becames normal as the means were taken from each
groups.}\label{comparing-the-first-with-the-second-histogram-we-can-see-the-distrubution-becames-normal-as-the-means-were-taken-from-each-groups.}}

\hypertarget{it-is-a-result-of-the-central-limit-theorem.}{%
\paragraph{4. It is a result of the central limit
theorem.}\label{it-is-a-result-of-the-central-limit-theorem.}}

\hypertarget{comparing-the-second-and-the-third-histogram-we-can-see-the-distribution-of-the-means-is-similar-to-a-real-normal-distribution-with-the-same-mean-and-standard-deviation.}{%
\paragraph{5. Comparing the second and the third histogram, we can see
the distribution of the means is similar to a real normal distribution
with the same mean and standard
deviation.}\label{comparing-the-second-and-the-third-histogram-we-can-see-the-distribution-of-the-means-is-similar-to-a-real-normal-distribution-with-the-same-mean-and-standard-deviation.}}

\hypertarget{part-2analyze-the-toothgrowth-data-in-the-r-datasets-package}{%
\section{Part 2:Analyze the ToothGrowth data in the R datasets
package}\label{part-2analyze-the-toothgrowth-data-in-the-r-datasets-package}}

\hypertarget{overview-in-this-part-we-will-do-some-statistical-data-analysises-about-the-toothlength-data.-question-1-load-the-toothgrowth-data-and-perform-some-basic-exploratory-data-analyses.}{%
\section{Overview: In this part we will do some statistical data
analysises about the Toothlength data. \#\#\#Question 1: Load the
ToothGrowth data and perform some basic exploratory data
analyses.}\label{overview-in-this-part-we-will-do-some-statistical-data-analysises-about-the-toothlength-data.-question-1-load-the-toothgrowth-data-and-perform-some-basic-exploratory-data-analyses.}}

library(stats) data(ToothGrowth) library(ggplot2) qplot(dose, len, data
= ToothGrowth, color = supp, geom = ``point'') + geom\_smooth(method =
``lm'') + labs(title = ``ToothGrowth'') + labs(x = ``Dose of
supplements'', y = ``Length of teeth'')

\hypertarget{from-the-plot-we-can-get}{%
\subsubsection{From the plot we can
get}\label{from-the-plot-we-can-get}}

\hypertarget{the-length-of-teeth-goes-up-as-the-dose-of-supplements-increases-which-indicates-that-the-supplements-may-help-teeth-growth.}{%
\paragraph{1. the length of teeth goes up as the dose of supplements
increases, which indicates that the supplements may help teeth
growth.}\label{the-length-of-teeth-goes-up-as-the-dose-of-supplements-increases-which-indicates-that-the-supplements-may-help-teeth-growth.}}

\hypertarget{at-the-same-dose-oj-seems-to-incur-a-higher-increase-of-teeth-growth-than-vc.}{%
\paragraph{2. At the same dose, OJ seems to incur a higher increase of
teeth growth than
VC.}\label{at-the-same-dose-oj-seems-to-incur-a-higher-increase-of-teeth-growth-than-vc.}}

\hypertarget{the-slope-of-oj-is-not-as-steep-as-the-slope-of-vc-meaning-an-increase-in-vc-may-make-a-larger-increase-in-teeth-length-than-in-oj.}{%
\paragraph{3. the slope of OJ is not as steep as the slope of VC,
meaning an increase in VC may make a larger increase in teeth length
than in
OJ.}\label{the-slope-of-oj-is-not-as-steep-as-the-slope-of-vc-meaning-an-increase-in-vc-may-make-a-larger-increase-in-teeth-length-than-in-oj.}}

\hypertarget{question-2-provide-a-basic-summary-of-the-data.}{%
\subsection{Question 2: Provide a basic summary of the
data.}\label{question-2-provide-a-basic-summary-of-the-data.}}

\hypertarget{overally-the-spread-of-the-length-data-is-as-follows-with-a-mean-of-18.81-and-a-standard-deviation-of-7.65.}{%
\subsubsection{Overally, the spread of the length data is as follows,
with a mean of 18.81 and a standard deviation of
7.65.}\label{overally-the-spread-of-the-length-data-is-as-follows-with-a-mean-of-18.81-and-a-standard-deviation-of-7.65.}}

summary(ToothGrowth)

sd(ToothGrowth\$len)

\hypertarget{the-summary-of-the-length-data-while-taking-oj-as-supplement-is-as-follows-with-a-mean-of-20.66-and-a-standard-deviation-of-6.61.}{%
\subsubsection{The summary of the length data while taking OJ as
supplement is as follows, with a mean of 20.66 and a standard deviation
of
6.61.}\label{the-summary-of-the-length-data-while-taking-oj-as-supplement-is-as-follows-with-a-mean-of-20.66-and-a-standard-deviation-of-6.61.}}

summary(ToothGrowth{[}ToothGrowth\$supp == ``OJ'', {]})

sd(ToothGrowth{[}ToothGrowth\(supp == "OJ", ]\)len)

\hypertarget{the-summary-of-the-length-data-while-taking-vc-as-supplement-is-as-follows-with-a-mean-of-16.96-and-a-standard-deviation-of-8.27.}{%
\subsubsection{The summary of the length data while taking VC as
supplement is as follows, with a mean of 16.96 and a standard deviation
of
8.27.}\label{the-summary-of-the-length-data-while-taking-vc-as-supplement-is-as-follows-with-a-mean-of-16.96-and-a-standard-deviation-of-8.27.}}

summary(ToothGrowth{[}ToothGrowth\$supp == ``VC'', {]})

sd(ToothGrowth{[}ToothGrowth\(supp == "VC", ]\)len)

\hypertarget{to-see-a-better-spread-of-the-data-we-may-look-at-the-boxplots-under-different-values-of-dose-and-different-kinds-of-supplements.}{%
\subsubsection{To see a better spread of the data, we may look at the
boxplots under different values of dose and different kinds of
supplements.}\label{to-see-a-better-spread-of-the-data-we-may-look-at-the-boxplots-under-different-values-of-dose-and-different-kinds-of-supplements.}}

g \textless- ggplot(ToothGrowth, aes(dose, len, group = supp)) g +
geom\_boxplot() + facet\_grid(supp \textasciitilde{} dose) + labs(x =
``Dose of supplements'', y = ``Length of teeth growth'')

\hypertarget{question-3-use-confidence-intervals-andor-hypothesis-tests-to-compare-tooth-growth-by-supp-and-dose.-only-use-the-techniques-from-class-even-if-theres-other-approaches-worth-considering}{%
\subsection{Question 3: Use confidence intervals and/or hypothesis tests
to compare tooth growth by supp and dose. (Only use the techniques from
class, even if there's other approaches worth
considering)}\label{question-3-use-confidence-intervals-andor-hypothesis-tests-to-compare-tooth-growth-by-supp-and-dose.-only-use-the-techniques-from-class-even-if-theres-other-approaches-worth-considering}}

\hypertarget{if-we-assume-the-data-is-normally-distributed-we-have-a-null-hypothesis-that-there-is-no-difference-between-the-mean-under-each-kind-of-supplements-or-each-dose-of-the-supplements}{%
\subsubsection{If we assume the data is normally distributed, we have a
null hypothesis that there is no difference between the mean under each
kind of supplements, or each dose of the
supplements:}\label{if-we-assume-the-data-is-normally-distributed-we-have-a-null-hypothesis-that-there-is-no-difference-between-the-mean-under-each-kind-of-supplements-or-each-dose-of-the-supplements}}

t.test(x =
ToothGrowth\(len, data = ToothGrowth, paired = FALSE, conf.level = 0.95)\)conf.

\hypertarget{we-will-be-able-to-construct-a-confidence-interval-that-95-of-the-time-an-interval-between-16.84-and-20.79-will-contain-the-true-mean-of-the-population.}{%
\subsubsection{We will be able to construct a confidence interval that
95\% of the time, an interval between 16.84 and 20.79 will contain the
true mean of the
population.}\label{we-will-be-able-to-construct-a-confidence-interval-that-95-of-the-time-an-interval-between-16.84-and-20.79-will-contain-the-true-mean-of-the-population.}}

\#\#\#Then we calculate the mean under each kind of supplements, and
each dose of the supplements:

summary(ToothGrowth{[}ToothGrowth\(supp == "OJ", ]\)len){[}4{]}

summary(ToothGrowth{[}ToothGrowth\(supp == "VC", ]\)len){[}4{]}

\hypertarget{thus-the-mean-of-teeth-growth-after-taking-oj-is-20.66-the-mean-of-teeth-growth-after-taking-vc-is-16.96.-both-the-two-are-within-the-confidence-interval.-we-fail-to-reject-the-null-hypothesis-that-there-is-not-a-difference-in-teeth-growth-after-taking-the-two-kinds-of-supplements.}{%
\subsubsection{Thus the mean of teeth growth after taking OJ is 20.66;
the mean of teeth growth after taking VC is 16.96. Both the two are
within the confidence interval. We fail to reject the null hypothesis
that there is not a difference in teeth growth after taking the two
kinds of
supplements.}\label{thus-the-mean-of-teeth-growth-after-taking-oj-is-20.66-the-mean-of-teeth-growth-after-taking-vc-is-16.96.-both-the-two-are-within-the-confidence-interval.-we-fail-to-reject-the-null-hypothesis-that-there-is-not-a-difference-in-teeth-growth-after-taking-the-two-kinds-of-supplements.}}

summary(ToothGrowth{[}ToothGrowth\(dose == 0.5, ]\)len){[}4{]}

summary(ToothGrowth{[}ToothGrowth\(dose == 1, ]\)len){[}4{]}

summary(ToothGrowth{[}ToothGrowth\(dose == 2, ]\)len){[}4{]}

\hypertarget{thus-the-mean-of-teeth-growth-after-taking-dose-of-0.5-is-10.6-the-mean-of-teeth-growth-after-taking-dose-of-1.0-is-19.74-the-mean-of-teeth-growth-after-taking-dose-of-2.0-is-26.1.-we-are-able-to-reject-the-null-hypotheis-and-there-is-a-difference-in-teeth-growth-between-each-dose-of-supplements.}{%
\paragraph{Thus the mean of teeth growth after taking dose of 0.5 is
10.6; the mean of teeth growth after taking dose of 1.0 is 19.74; the
mean of teeth growth after taking dose of 2.0 is 26.1. We are able to
reject the null hypotheis, and there is a difference in teeth growth
between each dose of
supplements.}\label{thus-the-mean-of-teeth-growth-after-taking-dose-of-0.5-is-10.6-the-mean-of-teeth-growth-after-taking-dose-of-1.0-is-19.74-the-mean-of-teeth-growth-after-taking-dose-of-2.0-is-26.1.-we-are-able-to-reject-the-null-hypotheis-and-there-is-a-difference-in-teeth-growth-between-each-dose-of-supplements.}}

\hypertarget{now-we-know-the-data-is-not-normally-distributed-under-each-dose-along-with-this-conclusion-we-may-assume-the-data-is-normally-distributed-within-each-dose.-then-we-will-be-able-to-compare-the-teeth-growth-between-each-supplements-under-each-dose}{%
\subsection{Now we know the data is not normally distributed under each
dose, along with this conclusion, we may assume the data is normally
distributed within each dose. Then we will be able to compare the teeth
growth between each supplements under each
dose}\label{now-we-know-the-data-is-not-normally-distributed-under-each-dose-along-with-this-conclusion-we-may-assume-the-data-is-normally-distributed-within-each-dose.-then-we-will-be-able-to-compare-the-teeth-growth-between-each-supplements-under-each-dose}}

dose05 \textless-
ToothGrowth{[}ToothGrowth\(dose == 0.5, ] t.test(x = dose05\)len, paired
= FALSE, conf.level = 0.95)\$conf.

mean(dose05{[}dose05\(supp == "VC", ]\)len)

mean(dose05{[}dose05\(supp == "OJ", ]\)len)

\hypertarget{thus-under-the-dose-of-0.5-there-are-95-of-the-time-that-a-confidence-interval-between-8.50-and-12.71-will-contain-the-true-population-mean.-we-also-know-that-the-mean-of-teeth-growth-after-taking-0.5-dose-of-vc-is-7.98-and-the-mean-of-teeth-growth-after-taking-0.5-dose-of-oj-is-13.23.-we-rejected-the-null-hypo.}{%
\subsubsection{Thus under the dose of 0.5, there are 95\% of the time
that a confidence interval between 8.50 and 12.71 will contain the true
population mean. We also know that the mean of teeth growth after taking
0.5 dose of VC is 7.98 and the mean of teeth growth after taking 0.5
dose of OJ is 13.23. We rejected the null
hypo.}\label{thus-under-the-dose-of-0.5-there-are-95-of-the-time-that-a-confidence-interval-between-8.50-and-12.71-will-contain-the-true-population-mean.-we-also-know-that-the-mean-of-teeth-growth-after-taking-0.5-dose-of-vc-is-7.98-and-the-mean-of-teeth-growth-after-taking-0.5-dose-of-oj-is-13.23.-we-rejected-the-null-hypo.}}

dose10 \textless-
ToothGrowth{[}ToothGrowth\(dose == 1, ] t.test(x = dose10\)len, paired =
FALSE, conf.level = 0.95)\$conf.

mean(dose10{[}dose10\(supp == "VC", ]\)len)

mean(dose10{[}dose10\(supp == "OJ", ]\)len)

\hypertarget{thus-under-the-dose-of-1.0-there-are-95-of-the-time-that-a-confidence-interval-between-17.67-and-21.80-will-contain-the-true-population-mean.-we-also-know-that-the-mean-of-teeth-growth-after-taking-1.0-dose-of-vc-is-16.77-and-the-mean-of-teeth-growth-after-taking-1.0-dose-of-oj-is-22.7.-we-rejected-the-null-hypo.}{%
\subsubsection{Thus under the dose of 1.0, there are 95\% of the time
that a confidence interval between 17.67 and 21.80 will contain the true
population mean. We also know that the mean of teeth growth after taking
1.0 dose of VC is 16.77 and the mean of teeth growth after taking 1.0
dose of OJ is 22.7. We rejected the null
hypo.}\label{thus-under-the-dose-of-1.0-there-are-95-of-the-time-that-a-confidence-interval-between-17.67-and-21.80-will-contain-the-true-population-mean.-we-also-know-that-the-mean-of-teeth-growth-after-taking-1.0-dose-of-vc-is-16.77-and-the-mean-of-teeth-growth-after-taking-1.0-dose-of-oj-is-22.7.-we-rejected-the-null-hypo.}}

dose20 \textless-
ToothGrowth{[}ToothGrowth\(dose == 2, ] t.test(x = dose20\)len, paired =
FALSE, conf.level = 0.95)\$conf.

mean(dose20{[}dose20\(supp == "VC", ]\)len)

mean(dose20{[}dose20\(supp == "OJ", ]\)len)

\hypertarget{thus-under-the-dose-of-2.0-there-are-95-of-the-time-that-a-confidence-interval-between-24.33-and-27.87-will-contain-the-true-population-mean.-we-also-know-that-the-mean-of-teeth-growth-after-taking-2.0-dose-of-vc-is-26.14-and-the-mean-of-teeth-growth-after-taking-2.0-dose-of-oj-is-26.06.-both-of-them-are-within-the-confidence-interval.-we-fail-to-reject-the-null-hypothesis.}{%
\subsubsection{Thus under the dose of 2.0, there are 95\% of the time
that a confidence interval between 24.33 and 27.87 will contain the true
population mean. We also know that the mean of teeth growth after taking
2.0 dose of VC is 26.14 and the mean of teeth growth after taking 2.0
dose of OJ is 26.06. Both of them are within the confidence interval. We
fail to reject the null
hypothesis.}\label{thus-under-the-dose-of-2.0-there-are-95-of-the-time-that-a-confidence-interval-between-24.33-and-27.87-will-contain-the-true-population-mean.-we-also-know-that-the-mean-of-teeth-growth-after-taking-2.0-dose-of-vc-is-26.14-and-the-mean-of-teeth-growth-after-taking-2.0-dose-of-oj-is-26.06.-both-of-them-are-within-the-confidence-interval.-we-fail-to-reject-the-null-hypothesis.}}

\hypertarget{question-4-state-your-conclusions-and-the-assumptions-needed-for-your-conclusions.}{%
\subsection{Question 4: State your conclusions and the assumptions
needed for your
conclusions.}\label{question-4-state-your-conclusions-and-the-assumptions-needed-for-your-conclusions.}}

\hypertarget{the-conclusion-is-when-the-dose-is-0.5-or-1.0-there-is-a-difference-between-the-teeth-growth-after-taking-oj-and-vc-while-when-the-dose-is-2.0-there-is-no-difference-between-the-teeth-growth-after-taking-oj-and-vc.-the-assumption-needed-is-we-first-assumed-the-whole-population-is-normally-distributed-then-we-assumed-the-population-is-normally-distributed-under-each-dose.}{%
\paragraph{The conclusion is when the dose is 0.5 or 1.0 there is a
difference between the teeth growth after taking OJ and VC, while when
the dose is 2.0, there is no difference between the teeth growth after
taking OJ and VC. The assumption needed is we first assumed the whole
population is normally distributed, then we assumed the population is
normally distributed under each
dose.}\label{the-conclusion-is-when-the-dose-is-0.5-or-1.0-there-is-a-difference-between-the-teeth-growth-after-taking-oj-and-vc-while-when-the-dose-is-2.0-there-is-no-difference-between-the-teeth-growth-after-taking-oj-and-vc.-the-assumption-needed-is-we-first-assumed-the-whole-population-is-normally-distributed-then-we-assumed-the-population-is-normally-distributed-under-each-dose.}}

\end{document}
